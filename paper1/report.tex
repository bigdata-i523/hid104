\documentclass[sigconf]{acmart}

\usepackage{hyperref}

\usepackage{endfloat}
\renewcommand{\efloatseparator}{\mbox{}} % no new page between figures

\usepackage{booktabs} % For formal tables

\settopmatter{printacmref=false} % Removes citation information below abstract
\renewcommand\footnotetextcopyrightpermission[1]{} % removes footnote with conference information in first column
\pagestyle{plain} % removes running headers

\begin{document}
\title{What Separates "Big Data" from "Lots of Data"}


\author{Gabriel Jones}
\orcid{1234-5678-9012}
\affiliation{%
  \institution{Indiana University}
  \streetaddress{107 S Indiana Ave}
  \city{Bloomington} 
  \state{Indiana}
  \country{USA}
  \postcode{47405}
}
\email{gabejone@indiana.edu}

% The default list of authors is too long for headers}
\renewcommand{\shortauthors}{G. Jones}


\begin{abstract}
TIn this paper, we will briefly analyze the history of data to show how having “lots of data” stored in large databases hardly differs from data storage and analysis in the early days of SQL, or even before computers. We then explain how “big data” represents a paradigmatic shift from traditional large data storage and analysis. We conclude that organizations that do not understand this paradigmatic shift are more likely to fail in big data projects.
\end{abstract}

\keywords{i523}


\maketitle

\section{Introduction}
This is my introduction.

\section{Conclusions}
I conclude that...



\begin{acks}

Generic acknowledgements

\end{acks}

\bibliographystyle{ACM-Reference-Format}
\bibliography{report} 

\end{document}
